\chapter{Campos finitos}
La aritmética en campos finitos es fundamental para numerosos algoritmos de clave pública, incluidos los basados en el problema del logaritmo discreto en campos finitos, los esquemas de curvas elípticas y las aplicaciones emergentes de curvas hiperelípticas. El rendimiento de estos protocolos depende en gran medida de nuestra capacidad para realizar rápidamente operaciones básicas en el campo subyacente.  

Los campos finitos (también llamados cuerpo finito o cuerpo de Galois) se denotan como \(\mathrm{GF}(p^m)\), donde \(p\) es un número primo y \(m\) un entero positivo. Para que los sistemas de curvas elípticas sean eficientes, es imprescindible implementar de manera óptima la suma, la resta, la multiplicación y la inversión en dicho campo. Existen tres familias de campos que resultan especialmente adecuadas para ECC (Criptografía de Curvas Elípticas):  
\begin{itemize}
  \item \textbf{Campos primos} (\(\mathbb{F}_p\)),  
  \item \textbf{Campos binarios} (\(\mathbb{F}_{2^m}\)),  
  \item \textbf{Campos de extensión óptima} (OEF).  
\end{itemize}
En las secciones siguientes se presenta una introducción informal a la teoría de campos finitos y se describen en detalle los algoritmos eficientes para cada tipo de campo.  

Dado que en muchas aplicaciones los tamaños de \(p^m\) son muy grandes, entender la complejidad computacional de estas operaciones es clave para evaluar el rendimiento global de los esquemas criptográficos basados en curvas elípticas.  

Nota: $GF(p^m)$ == \(\mathbb{F}_{p^m}\).

\section{Introducción a campos finitos}
Los campos son abstracciones de sistemas numéricos familiares (como los racionales \(\mathbb{Q}\), los reales \(\mathbb{R}\) o los complejos \(\mathbb{C}\)) que reúnen ciertas propiedades esenciales. Un campo \(\mathbb{F}\) consta de un conjunto \(\mathbb{F}\) junto con dos operaciones, suma (\(+\)) y producto (\(\cdot\)), que satisfacen:

\begin{enumerate}
  \item $(\mathbb{F},+)$ es un grupo abeliano con identidad suma denotada por $0$.
  \item $(\mathbb{F}\setminus\{0\},\cdot)$ es un grupo abeliano con identidad multiplicativa denotada por $1$.
  \item La multiplicación distribuye sobre la suma:
    \[
      (a + b)\cdot c = a\cdot c + b\cdot c
      \quad\text{para todo }a,b,c\in\mathbb{F}.
    \]
\end{enumerate}

Si el conjunto \(\mathbb{F}\) es finito, se dice que el campo es finito.  

En un campo, la resta se define mediante la suma de inversos aditivos: para \(a,b\in\mathbb{F}\),  
\[
  a - b \;=\; a + (-b),
\]
donde \(-b\) es el elemento único que satisface \(b + (-b) = 0\).  

La división se define usando el inverso multiplicativo: para \(a,b\in\mathbb{F}\) con \(b\neq0\),
\[
  \frac{a}{b} \;=\; a \cdot b^{-1},
\]
siendo \(b^{-1}\) el único elemento que cumple \(b\cdot b^{-1}=1\).  

El \emph{orden} de un campo finito es el número de sus elementos. Existe un campo finito de orden \(q\) si y solo si \(q\) es una potencia prima, es decir, \(q = p^m\) con \(p\) primo (característica del campo) y \(m\) entero positivo.  

\begin{itemize}
  \item Si \(m=1\), el campo se llama \emph{campo primo} y se denota \(\mathbb{F}_p\).  
  \item Si \(m\ge2\), se llama \emph{campo de extensión} y se denota \(\mathbb{F}_{p^m}\).  
\end{itemize}

Para cada potencia prima \(q\), existe esencialmente un único campo finito de orden \(q\): todos ellos son isomorfos (idénticos en estructura, aunque difiera la representación de sus elementos). Por ello se usa la notación general \(\mathbb{F}_q\).  



\section{Aritmética en \texorpdfstring{$\mathbb{F}_p$}{Fp}}
Sea $p$ un número primo. El conjunto
$$
  \mathbb{F}_p = \{0,1,2,\dots,p-1\},
$$
con las operaciones de suma y multiplicación módulo $p$, $\mathbb{Z}/p\mathbb{Z}$, constituye un campo finito de orden $p$. Denotaremos este campo por $\mathbb{F}_p$ y llamaremos a $p$ el \emph{módulo} de $\mathbb{F}_p$. Para cualquier entero $a$, $a \bmod p$ denota el único residuo $r$, $0\le r\le p-1$, obtenido al dividir $a$ por $p$; esta operación se llama \emph{reducción módulo $p$}.

Todos los elementos no nulos de $\mathbb{F}_p$ tienen inverso multiplicativo, y forman un grupo cíclico de orden $p-1$ (esto es una consecuencia del pequeño Teorema de Fermat).

\paragraph{Ejemplo 2.1 (campo primo $\mathbb{F}_{31}$)}  
Los elementos de $\mathbb{F}_{31}$ son $\{0,1,2,\dots,30\}$. A continuación se muestran algunas operaciones en $\mathbb{F}_{31}$:

\begin{enumerate}
  \item \textbf{Suma:} $9 + 25 = 34 \equiv 3 \pmod{31}.$
  \item \textbf{Resta:} $9 - 25 = -16 \equiv 15 \pmod{31}.$
  \item \textbf{Multiplicación:} $9 \cdot 25 = 225 \equiv 8 \pmod{31}.$
  \item \textbf{Inverso multiplicativo:} $9^{-1} = 7$ ya que $9 \cdot 7 = 63 \equiv 1 \pmod{31}.$
\end{enumerate}

\section{Aritmética en \texorpdfstring{$\mathbb{F}_{2^m}$}{F2m}}
Los campos finitos de orden $2^m$, también llamados \emph{campos binarios}, se construyen como extensiones de polinomios:
\[
  \mathbb{F}_{2^m}\cong\mathbb{F}_2[x]/(f(x)),
\]
donde $f(x)$ es un polinomio irreducible de grado $m$ sobre $\mathbb{F}_2$.
En tal campo, cada elemento puede representarse como un polinomio de grado < $m$ con coeficientes en ${0,1}$ (equivalentemente, como un vector binario de $m$ bits).

\subsection{Representación polinomial}
Cada elemento se escribe como
\[
  a(x)=a_{m-1}x^{m-1}+\cdots+a_1x+a_0,\quad a_i\in\{0,1\}.
\]
La suma de elementos se realiza como la suma de polinomios coeficiente a coeficiente (operación equivalente a XOR bit a bit, sin acarreo), y la multiplicación se reduce módulo $f(x)$:
\[
  a(x)\cdot b(x)\bmod f(x).
\]
La reducción $p(x)\bmod f(x)$ es el residuo de grado < $m$ tras la división larga.

\paragraph{Ejemplo 2.2 (campo $\mathbb{F}_{2^3}$)}
Sea $f(x)=x^3+x+1$. Entonces
\[
  \mathbb{F}_{2^3}=\{0,1,x,x+1,x^2,x^2+1,x^2+x,x^2+x+1\}.
\]
\begin{enumerate}
  \item Suma: $(x^2+x+1)+(x+1)=x^2$.
  \item Multiplicación: $(x^2+x+1)(x+1)=x^3+x+1\equiv x+1\pmod{f(x)}$.
  \item Inverso: $(x^2+x+1)^{-1}=x^2$, pues $(x^2+x+1)x^2=x^4+x^3+x^2\equiv1\pmod{f(x)}$.
\end{enumerate}

\paragraph{Ejemplo 2.3 (isomorfismo de campos)}
Para $m=3$ hay dos irreducibles: $f_1(x)=x^3+x+1$ y $f_2(x)=x^3+x^2+1$. Los campos $\mathbb{F}_2[x]/(f_1)$ y $\mathbb{F}_2[x]/(f_2)$ tienen idénticos elementos y son isomorfos, pues existe un mapeo que envía la clase de $x$ en uno a una raíz de $f_1$ en el otro.

\section{Campos de extensión}
La representación por base polinomial para campos binarios se generaliza a todos los campos de extensión de la siguiente manera. Sea $p$ un primo y $m\ge2$. Denotamos por $\mathbb{F}_p[z]$ el anillo de polinomios en la variable $z$ con coeficientes en $\mathbb{F}_p$. Sea $f(z)\in\mathbb{F}_p[z]$ un polinomio irreducible de grado $m$—existen para cualquier par $(p,m)$ y pueden hallarse eficientemente (ver Apéndice A). La irreducibilidad significa que $f(z)$ no se factoriza como producto de polinomios en $\mathbb{F}_p[z]$ de grado menor que $m$. Entonces:
\[
  \mathbb{F}_{p^m}=\{a_{m-1}z^{m-1}+\cdots+a_1z+a_0:\,a_i\in\mathbb{F}_p\}
\]
con suma habitual de polinomios y producto reducido módulo $f(z)$.

\paragraph{Ejemplo 2.4 (campo $\mathbb{F}_{251^5}$)}
Sea $p=251$, $m=5$ y $f(z)=z^5+z^4+12z^3+9z^2+7$, irreducible en $\mathbb{F}_{251}[z]$. Entonces $\mathbb{F}_{251^5}$ consta de polinomios de grado \(<5\) con coeficientes en $\{0,1,\dots,250\}$. Por ejemplo, sean:
\[
  a=123z^4+76z^2+7z+4,\quad b=196z^4+12z^3+225z^2+76.
\]
\begin{enumerate}
  \item Suma: $a+b=68z^4+12z^3+50z^2+7z+80.$
  \item Resta: $a-b=178z^4+239z^3+102z^2+7z+179.$
  \item Multiplicación: $a\cdot b=117z^4+151z^3+117z^2+182z+217.$
  \item Inverso: $a^{-1}=109z^4+111z^3+250z^2+98z+85,$
\end{enumerate}
obeylines

\section{Grupos multiplicativos y orden de un elemento}
En todo campo finito $\mathbb{F}_q$, el conjunto $\mathbb{F}_q^*=\mathbb{F}_q\setminus\{0\}$ es un grupo cíclico de orden $q-1$. Un generador $g$ (elemento primitivo) cumple:
\[
  \langle g\rangle=\mathbb{F}_q^*,\quad g^k\neq1\text{ para }0<k<q-1.
\]
El \emph{orden} de $a\in\mathbb{F}_q^*$ es el menor $d>0$ tal que $a^d=1$. Determinarlo es clave al definir curvas elípticas seguras.
