\chapter{Introducción a la criptografía}
La criptografía es el arte de encubrir los mensajes con signos
convencionales, que sólo pueden cobrar algún sentido a través de una clave
secreta que nace en conjunto con la escritura.

En su clasificación dentro de las ciencias, la criptografía proviene de una rama de las matemáticas, que fue
iniciada por el matemático Claude Elwood Shannon en 1948, denominada: "Teoría de la Información".
Esta rama de las ciencias se divide en: "Teoría de Códigos" y en "Criptología". Y a su vez la Criptología se divide en Criptoanálisis y Criptografía, como se muestra en la siguiente figura~\ref{fig:esquema_criptografia}:
\begin{figure}[H]
    \centering
    \includegraphics[width=0.6\textwidth]{imagenes/Esquema_criptografía.png}
    \caption{Esquema de clasificación dentro de la Teoría de la Información}
    \label{fig:esquema_criptografia}
\end{figure}

En un sentido más amplio, la Criptografía es la ciencia encargada de diseñar funciones o dispositivos, capaces
de transformar mensajes legibles o en claro a mensajes cifrados de tal manera que esta transformación (cifrar) y su transformación inversa (descifrar) sólo pueden ser factibles con el conocimiento de una o más llaves.
En contraparte, el criptoanálisis es la ciencia que estudia los métodos que se utilizan para, a partir de uno o varios mensajes cifrados, recuperar los mensajes en claro en ausencia de la(s) llave(s) y/o encontrar la llave o llaves con las que fueron cifrados dichos mensajes.

La criptografía se puede clasificar históricamente en dos:
\begin{itemize}
    \item Criptografía Clásica: es aquella que se utilizó desde antes de la época actual hasta la mitad del siglo XX. También puede entenderse como la criptografía no computarizada o mejor dicho no digitalizada. Los métodos utilizados eran variados, algunos muy simples y otros muy complicados de criptoanalizar para su época. 
    \begin{itemize}
        \item \textbf{Sustitución:} Se reemplazan elementos del mensaje original por otros.
        \item \textbf{Transposición:} Se reordenan los caracteres del mensaje siguiendo un patrón definido.
    \end{itemize}
    
    \item Criptografía Moderna: se divide actualmente en cifrado sin clave (funciones hash), cifrado de clave privada (cifrado simétrico) y cifrado de clave pública (cifrado asimétrico). En el cifrado de clave privada las claves de cifrado y descifrado son la misma (o bien se deriva de forma directa una de la otra), debiendo mantenerse en secreto dicha clave para evitar el descifrado de un mensaje. Por el contrario, en el cifrado de clave pública las claves de cifrado y descifrado son independientes, no derivándose una de la otra, por lo cual puede hacerse pública la clave de cifrado siempre que se mantenga en secreto la clave de descifrado. Veremos algunos algoritmos de cifrado modernos
\end{itemize}

Tanto la criptografía clásica como la moderna se clasifican de acuerdo a las técnicas o métodos que se utilizan para cifrar los mensajes. Esta clasificación la podemos ver en la siguiente figura~\ref{fig:clasificacion_criptografia}:

\begin{figure}[H]
    \centering
    \includegraphics[width=0.8\textwidth]{imagenes/Clasificación_criptografía.png}
    \caption{Clasificación de la criptografía en función del método de cifrado}
    \label{fig:clasificacion_criptografia}
\end{figure}

\section{Estructura de un criptosistema}

Un criptosistema se define formalmente como la tupla
\[
  \bigl(\mathcal M,\;\mathcal C,\;\mathcal K,\;\mathcal E,\;\mathcal D\bigr),
\]
donde
\begin{itemize}
  \item \(\mathcal M\) es el espacio de texto plano.
  \item \(\mathcal C\) es el espacio de texto cifrado.
  \item \(\mathcal K\) es el espacio de claves.
  \item \(\mathcal E=\{E_k: k\in\mathcal K\}\) es el conjunto de funciones de cifrado \(E_k:\mathcal M\to\mathcal C\).
  \item \(\mathcal D=\{D_k: k\in\mathcal K\}\) es el conjunto de funciones de descifrado \(D_k:\mathcal C\to\mathcal M\).
\end{itemize}
Deben cumplirse las condiciones de corrección:
\[
  D_{k'}\bigl(E_k(M)\bigr)\;=\;M
  \quad\forall\,M\in\mathcal M,
\]
siendo \(k'\) la clave de descifrado asociada a la clave de cifrado \(k\).

\begin{figure}[H]
    \centering
    \includegraphics[width=0.8\textwidth]{imagenes/Esquema_criptosistema_cifrado_descifrado.png}
    \caption{Esquema de un criptosistema formal}
    \label{fig:esquema_criptosistema_cifrado_descifrado}
\end{figure}

\section{Criptografía sin clave o Funciones hash}
Las funciones hash son funciones unidireccionales de cálculo sencillo y eficiente, utilizadas para garantizar la integridad de los datos. Se aplican a un mensaje \(M\) produciendo un valor de longitud fija:
\[
  M \;\longmapsto\; H(M),
\]
donde \(H(M)\) suele tener 128, 160, 256 o 512 bits. Es computacionalmente inviable recuperar \(M\) a partir de \(H(M)\).

Una función hash criptográfica es una función eficiente que mapea cadenas binarias de longitud arbitraria a cadenas binarias de longitud fija \(n\). Para una función que produce valores de \(n\) bits, la probabilidad de que una cadena aleatoria termine en un valor dado es \(2^{-n}\). El valor de hash actúa como representante compacto de la cadena de entrada.

Para su uso en criptografía, \(H\) debe satisfacer al menos:

\begin{itemize}
  \item \textbf{Resistencia a preimagen:} Dado un valor \(y\) de \(n\) bits, resulta inviable encontrar \(x\) tal que \(H(x)=y\).
  \item \textbf{Resistencia a segunda preimagen:} Dado un mensaje \(x\), resulta inviable hallar otro \(x'\neq x\) con \(H(x')=H(x)\).
  \item \textbf{Resistencia a colisiones:} Resulta inviable encontrar dos entradas distintas \(x,y\) tales que \(H(x)=H(y)\).
\end{itemize}

Cuando \(H\) es público y sin clave, se le llama \emph{código de detección de modificaciones} (MDC). Si se añade una clave secreta al cómputo de hash, obtenemos un \emph{código de autenticación de mensaje} (MAC), que ofrece además autenticación del origen junto con integridad.

\subsection{MD5}
El MD5 (Message-Digest Algorithm 5) procesa el mensaje de entrada en bloques sucesivos de 512 bits y produce un resumen (digest) de 128 bits~\citep{IETF1321}.  
Su diseño sigue la construcción de Merkle–Damgard, inicializando cuatro registros de 32 bits y aplicando cuatro rondas de 16 operaciones cada una por bloque~\citep{Wang04_MD5}.  
En 2004 se demostró que MD5 ya no es resistente a colisiones: dos entradas diferentes pueden colisionar y producir el mismo digest en tiempo práctico~\citep{Wang04_MD5}.  
Por ejemplo, en 2009, Marc Stevens, Arjen Lenstra y Benne de Weger, un grupo de investigadores, construyeron colisiones con prefijos elegidos y generaron un certificado CA malicioso~\citep{Stevens09_MD5_CA}.  
Actualmente se desaconseja MD5 para cualquier uso criptográfico y se recomienda SHA-2 o SHA-3 como alternativas seguras~\citep{NIST_SP800131A}.


\subsection{SHA-1}
El SHA (Secure Hash Algorithm 1), fue publicado en 1995 como FIPS 180-1 por el NIST~\citep{FIPS1801}. Procesa los datos de entrada en bloques de 512 bits al igual que MD5, pero produce un digest de 160 bits con cinco registros de 32 bits y 80 rondas de mezcla~\citep{FIPS1801}. Su algoritmo es el siguiente:
\begin{enumerate}
    \item Se toma el mensaje original y se rellena hasta alcanzar una longitud de $448 \mod 512$ bits. Es decir, al dividir la longitud total entre 512, el resto debe ser 448.
    \item Se añade un campo de 64 bits que indica la longitud del mensaje original (incluyendo el propio relleno). Así, la longitud total del mensaje resultante, denotado $P'$, será un múltiplo de 512 bits.
    \item $P'$ se divide en bloques de longitud fija de 512 bits, denotados como $Y_1, Y_2, \ldots, Y_L$.
    \item Se inicia el cálculo del resumen. Cada bloque de 512 bits se mezcla con un búfer de 160 bits mediante 80 rondas. Cada 20 rondas se modifican las funciones de mezcla utilizadas. Este proceso continúa hasta procesar todos los bloques de entrada.
\end{enumerate}

Una vez terminado el cálculo, el buffer de 160 bits contiene el valor del compendio de mensaje. El SHA es $2^{32}$ veces más seguro que el MD5, pero su cálculo es más lento.

En 2005 Xiaoyun Wang y colaboradores redujeron la complejidad para hallar colisiones de \(2^{80}\) a \(2^{63}\) operaciones~\citep{Wang05_SHA1}, y en 2017 Google y CWI demostraron la primera colisión real de SHA-1 en el experimento \emph{SHAttered}~\citep{GoogleShattered}. El NIST anunció en 2011 la deprecación de SHA-1 y ha planificado retirarla completamente para fines de 2030~\citep{NIST_SP800131A}. Chrome dejó de aceptar certificados TLS firmados con SHA-1 en enero de 2017~\citep{ChromeDeprecateSHA1}, Mozilla Firefox siguió en febrero de 2017~\citep{MozillaDeprecateSHA1} y Microsoft finalizó el soporte para contenidos firmados con SHA-1 en 2021~\citep{MicrosoftSHA1Deprecation}.

\section{Criptografía de clave privada o simétrica}
Un cifrado simétrico utiliza una clave $k$ elegida de un espacio (conjunto) de claves posibles $K$ para cifrar un mensaje $m$ perteneciente al conjunto de mensajes posibles $M$, generando un texto cifrado $c \in C$.

El cifrado puede verse como una función:
\[
e : K \times m \rightarrow C
\]
cuyo dominio es el conjunto de pares $(k, m)$ y cuyo codominio es el espacio de textos cifrados $C$.

El descifrado se representa como una función:
\[
d : K \times C \rightarrow m
\]

Queremos que el descifrado revierta el cifrado:
\[
d(k, e(k, m)) = m \quad \forall k \in K, \forall m \in M
\]

Usando notación de subíndice para la clave, podemos escribir:
\[
e_k : m \rightarrow C \quad \text{y} \quad d_k : C \rightarrow m
\]
cumpliendo:
\[
d_k(e_k(m)) = m \quad \forall m \in M
\]

Esto implica que $e_k$ es una función inyectiva. Es decir, si $e_k(m) = e_k(m')$, entonces:
\[
m = d_k(e_k(m)) = d_k(e_k(m')) = m' \quad \forall k \in K, \forall m \in M
\]

\subsection*{Propiedades deseables de un cifrado simétrico}

Sea $(K, m, C, e, d)$ un sistema de cifrado. Este debe cumplir las siguientes propiedades:

\begin{enumerate}
    \item Para todo $k \in K$ y $m \in M$, debe ser fácil computar $e_k(m)$.
    \item Para todo $k \in K$ y $c \in C$, debe ser fácil computar $d_k(c)$.
    \item Dados uno o más textos cifrados $c_1, \ldots, c_n$ producidos con una misma clave $k$, debe ser computacionalmente difícil recuperar $d_k(c_1), \ldots, d_k(c_n)$ sin conocer $k$.
\end{enumerate}

Con este tipo de criptografía podemos garantizar la confidencialidad porque únicamente quien posea la llave secreta será capaz de ver el mensaje.
El problema con la criptografía simétrica es que si yo quisiera compartir secretos con n personas, para cada persona tendría que generar una nueva llave secreta.
Otro problema asociado con este tipo de criptografía es cómo comparto con otra persona de una forma confidencial e integra la llave secreta.
Estos problemas se resuelven de cierta manera con criptografía asimétrica.

\subsection{Cifrado de bloque}
Los cifrados de bloque cifran mensajes dividiéndolos en grupos de símbolos del mismo tamaño (bloques) y aplicando sobre cada uno de ellos un mismo algoritmo de cifra.

Llamamos \(B\) el tamaño de bloque del cifrado. Un mensaje de texto claro general consiste entonces en una lista de bloques de mensaje escogidos de \(M\), y la función de cifrado transforma estos bloques en una lista de bloques de texto cifrado en \(C\), donde cada bloque es una secuencia de \(B\) bits. Si el texto claro termina con un bloque de menos de \(B\) bits, rellenamos el final del bloque con ceros.

El cifrado y el descifrado se realizan bloque a bloque, por lo que basta estudiar el proceso para un único bloque de texto claro, es decir, para un solo \(m\in M\). Esto, por supuesto, explica la conveniencia de dividir un mensaje en bloques: un mensaje puede ser de longitud arbitraria, y resulta útil centrar el proceso criptográfico en una pieza de longitud fija. El bloque de texto claro \(m\) es una cadena de \(B\) bits, que para mayor concreción identificamos con el número correspondiente en forma binaria. En otras palabras, identificamos \(M\) con el conjunto de enteros \(m\) que satisfacen \(0 \le m < 2^B\) mediante
\[
\underbrace{m_{B-1}\,m_{B-2}\,\cdots\,m_2\,m_1\,m_0}_{\text{lista de \(B\) bits de \(m\)}}
\]

\[
\longleftrightarrow
\]

\[
\underbrace{m_{B-1}\cdot2^{B-1} + m_{B-2}\cdot2^{B-2} + \cdots + m_1\cdot2 + m_0}_{\text{entero entre 0 y \(2^B-1\)}}
\]

Aquí \(m_0,m_1,\dots,m_{B-1}\) son cada uno 0 o 1.

De manera similar, identificamos el espacio de claves \(K\) y el espacio de textos cifrados \(C\) con conjuntos de enteros correspondientes a cadenas de bits de un determinado tamaño de bloque. Para mayor simplicidad notacional, denotamos los tamaños de bloque para claves, textos claros y textos cifrados por \(B_k\), \(B_m\) y \(B_c\), respectivamente. No tienen por qué coincidir. Así, hemos identificado
\[
K = \{\,k \in \mathbb{Z} : 0 \le k < 2^{B_k}\}
\]

\[
M = \{\,m \in \mathbb{Z} : 0 \le m < 2^{B_m}\}
\]

\[
C = \{\,c \in \mathbb{Z} : 0 \le c < 2^{B_c}\}
\]


Surge de inmediato una pregunta importante: ¿qué tan grande debe ser el conjunto \(K\) que se debe de escoger, equivalentemente, qué tamaño de bloque de clave \(B_k\) se debería de elegir? Si \(B_k\) es demasiado pequeño, se puede probar cada número entre \(0\) y \(2^{B_k}-1\) hasta dar con la clave generada. Más precisamente, dado que se asume que se conoce el algoritmo de descifrado \(d\) (principio de Kerckhoffs), toma cada \(k\in K\) y calcula \(d_k(c)\). Suponiendo que se es capaz de distinguir entre textos claros válidos e inválidos, se acabará recuperando el mensaje.

Este ataque se conoce como ataque de búsqueda exhaustiva (o ataque de fuerza bruta), puesto que se explora exhaustivamente el espacio de claves. Con la tecnología actual, una búsqueda exhaustiva se considera inviable si el espacio tiene al menos \(2^{80}\) elementos. Por tanto, se debería de  escoger \(B_k \ge 80\).

Para muchos criptosistemas, existen refinamientos al ataque de búsqueda exhaustiva que sustituyen efectivamente el tamaño del espacio por su raíz cuadrada. Estos métodos se basan en el principio de que es más fácil encontrar objetos coincidentes (colisiones) en un conjunto que localizar un objeto en particular. Describimos algunos de estos ataques de "man-in-the-middle" o de colisiones en la sección 2.6. Si están disponibles ataques de este tipo, se deberá de escoger \(B_k \ge 160\).


Existen distintos modos de funcionamiento para los cifrados de bloques. El \emph{Electronic CodeBook} (ECB) y el \emph{Cipher-Block Chanining} (CBC) son dos de los modos más antiguos. Entre los sistemas de cifrado por bloques más conocidos están el Data Encryption Standard, Triple DES (3DES)\citep{tdes-techtarget}, Advanced Encryption Standard (AES)\citep{aes-nist} y Twofish\citep{Block_Cipher}\citep{twofish-schneier}.


\subsubsection{ECB}
El \emph{Electronic CodeBook} (ECB) es el modo de funcionamiento más sencillo del cifrado por bloques. Es más fácil porque se cifra directamente cada bloque de texto plano de entrada y la salida es en forma de bloques de texto cifrado. Generalmente, si un mensaje tiene un tamaño superior a $b$ bits, se puede dividir en un montón de bloques y repetir el procedimiento. En la figura se muestra un ejemplo de funcionamiento del mismo~\ref{fig:ECB_block_cypher}
\begin{figure}[H]
    \centering
    \includegraphics[width=0.8\textwidth]{imagenes/ECB_block_cypher.png}
    \caption{Ejemplo de cifrado y descifrado de bloque ECB}
    \label{fig:ECB_block_cypher}
\end{figure}

\subsubsection{CBC}
El modo \emph{Cipher Block Chaining} (CBC) mejora la seguridad del modo ECB al encadenar cada bloque cifrado con el siguiente. 
Antes de entrar en su explicación, en este modo se introduce un nuevo concepto:
Un IV es esencialmente otro input (además del texto plano $M$ y la llave $K$) usado para cifrar el texto. Es un bloque de datos, usado por varios modos de cifrado de bloques para introducir aleatoriedad en el proceso de cifrado para poder generar como resultado distintos textos cifrados incluso con el mismo texto plano cifrado varias veces.

No es necesario que sea secreto, pero si es necesario que no se vuelva a usar. Es recomendable que sea aleatorio, impredicible y de un único uso.
De forma simplificada:
\begin{itemize}
  \item Para cada bloque $M_i$ de texto claro:
    \[
      C_i = E\bigl(M_i \;\oplus\; C_{i-1},\,K\bigr)
    \]
    donde $C_{i-1}$ es el bloque cifrado anterior.
  \item Para el primer bloque, no existe $C_0$, por lo que se usa un vector de inicialización (IV) aleatorio:
    \[
      C_1 = E\bigl(M_1 \;\oplus\; \mathrm{IV},\,K\bigr).
    \]
  \item De este modo, incluso si $M_i = M_j$, normalmente $C_i \neq C_j$ porque $C_{i-1}\neq C_{j-1}$.
\end{itemize}

\subsection*{Pasos de cifrado}
\begin{enumerate}
  \item Calcular $X_1 = M_1 \oplus \mathrm{IV}$ y luego $C_1 = E(X_1, K)$.
  \item Para $i > 1$, calcular $X_i = M_i \oplus C_{i-1}$ y luego $C_i = E(X_i, K)$.
  \item Repetir hasta procesar todos los bloques de texto claro.
\end{enumerate}

\subsection*{Pasos de descifrado}
\begin{enumerate}
  \item Descifrar $D_1 = D(C_1, K)$ y obtener $M_1 = D_1 \oplus \mathrm{IV}$.
  \item Para $i > 1$, descifrar $D_i = D(C_i, K)$ y obtener $M_i = D_i \oplus C_{i-1}$.
  \item Repetir hasta recuperar todos los bloques de texto original.
\end{enumerate}

En la figura se muestra un ejemplo de funcionamiento del mismo~\ref{fig:CBC_block_cypher}
\begin{figure}[H]
    \centering
    \includegraphics[width=0.8\textwidth]{imagenes/CBC_block_cypher.png}
    \caption{Ejemplo de cifrado y descifrado de bloque CBC}
    \label{fig:CBC_block_cypher}
\end{figure}

Una desventaja del mismo es que si se reutiliza el mismo IV para dos mensajes idénticos, se obtendrán los mismos bloques $C_1$, lo que compromete seguridad\cite{Block_Cipher}.

\subsection{Cifrado de flujo}
Los cifrados de flujo son ciertamente cifrados de bloque con una longitud de bloque igual a uno. Lo que los hace útiles es que la transformación de cifrado puede cambiar para cada símbolo del texto plano que se cifra.También pueden usarse cuando los datos deben procesarse símbolo por símbolo (por ejemplo, si el equipo no dispone de memoria o el almacenamiento intermedio es limitado).

Sea $K$ el espacio de claves. Una secuencia de símbolos $e_1e_2e_3\ldots \in K$ se denomina \textbf{flujo de claves} o \textit{keystream}.

Sea $A$ un alfabeto de $q$ símbolos, y sea $E_e$ un cifrado por sustitución simple con longitud de bloque 1, donde $e \in K$. Dado un texto claro $m_1m_2m_3\ldots$ y un flujo de claves $e_1e_2e_3\ldots$, un cifrado de flujo produce un texto cifrado $c_1c_2c_3\ldots$ tal que:

\[
c_i = E_{e_i}(m_i)
\]

Si $d_i$ denota la función inversa de $e_i$, entonces:

\[
D_{d_i}(c_i) = m_i
\]

Este tipo de criptografía se basa en hacer un cifrado bit a bit, esto se logra usando la operación XOR. Se utiliza un algoritmo determinístico que genera una secuencia pseudoaletoria de bits que junto con los bits del mensaje se van cifrando utilizando la operación XOR.

Un ejemplo de ello es el cifrado de vernam~\citep{vernam}, definido sobre $A = \{0, 1\}$. Dado un mensaje binario $m_1m_2\ldots m_t$ y una clave binaria $k_1k_2\ldots k_t$ de la misma longitud, el texto cifrado es:

\[
c_i = m_i \oplus k_i, \quad 1 \leq i \leq t
\]

Si la clave es aleatoria y no se reutiliza, se denomina \textbf{one-time pad}.

\textbf{Observación:} Hay dos cifrados por sustitución posibles sobre $A$:
\begin{itemize}
    \item $E_0$: $0 \mapsto 0$, $1 \mapsto 1$
    \item $E_1$: $0 \mapsto 1$, $1 \mapsto 0$
\end{itemize}

El flujo de claves determina cuál se aplica a cada símbolo.

Algunos ejemplos de cifrado de flujo son RC4 (usado en redes inalámbricas)\cite{rc4-ietf,rc4-gfg} y A5 (usado en
telefonía celular)\cite{a5-nop,a5-ucsb}.

\section{Criptografía de clave pública o asimétrica}
Si Alice y Bob quieren intercambiar mensajes usando un cifrado simétrico, deben primero acordar mutuamente una clave secreta \(k\). Esto es factible si tienen la oportunidad de reunirse en secreto o si pueden comunicarse una vez por un canal seguro. Pero ¿qué ocurre si no disponen de esa oportunidad y cada comunicación entre ellos es vigilada por su adversaria Eva? ¿Es posible que Alice y Bob intercambien una clave secreta en estas condiciones?

La primera reacción de la mayoría es que no es posible, ya que el intruso ve cada fragmento de información que intercambian. Fue la brillante intuición de Diffie y Hellman que, bajo ciertas hipótesis, sí lo es. La búsqueda de soluciones eficientes (y demostrables) a este problema, denominado criptografía de clave pública (o asimétrica), constituye una de las partes más interesantes de la criptografía matemática.

Definimos espacios de claves \(K\), de textos claros \(M\) y de textos cifrados \(C\). Sin embargo, un elemento \(k\) del espacio de claves es en realidad un par de claves,
\[
k = \bigl(k_{\text{priv}},\,k_{\text{pub}}\bigr),
\]
denominadas clave privada y clave pública, respectivamente. Para cada clave pública \(k_{\text{pub}}\) existe una función de cifrado correspondiente
\[
e_{k_{\text{pub}}}: M \longrightarrow C,
\]
y para cada clave privada \(k_{\text{priv}}\) existe una función de descifrado correspondiente
\[
d_{k_{\text{priv}}}: C \longrightarrow M.
\]
Estas funciones cumplen que si el par \(\bigl(k_{\text{priv}},\,k_{\text{pub}}\bigr)\) pertenece al espacio de claves \(K\), entonces
\[
d_{k_{\text{priv}}}\bigl(e_{k_{\text{pub}}}(m)\bigr) = m
\quad\forall\,m \in M.
\]

Para que un cifrado asimétrico sea seguro, debe resultar difícil para el intruso calcular la función de descifrado \(d_{k_{\text{priv}}}(c)\), incluso si conoce la clave pública \(k_{\text{pub}}\). Bajo esta suposición, Alice puede enviar \(k_{\text{pub}}\) a Bob mediante un canal inseguro, y Bob puede devolver el texto cifrado \(e_{k_{\text{pub}}}(m)\) sin temor a que el intruso lo descifre. Para poder descifrar, es necesario conocer la clave privada \(k_{\text{priv}}\), y, presumiblemente, solo Alice dispone de ella. A esta clave privada a veces se le llama información de \emph{trapdoor} de Alice, porque proporciona una vía directa para calcular la función inversa de \(e_{k_{\text{pub}}}\). El hecho de que las claves de cifrado y descifrado (\(k_{\text{pub}}\) y \(k_{\text{priv}}\)) sean diferentes confiere al cifrado su carácter asimétrico, de ahí su nombre.

Resulta fascinante que Diffie y Hellman concibieran este concepto sin contar con un par concreto de funciones; no obstante, sí propusieron un método análogo por el cual Alice y Bob pueden intercambiar de forma segura un dato aleatorio cuyo valor no conocen inicialmente. Describimos el método de intercambio de claves de Diffie y Hellman en la Sección ~\ref{sec:diffie_hellman} y, a continuación, analizaremos más adelante varios cifrados asimétricos como RSA (Sección~) y ECC (Sección~), cuya seguridad se fundamenta en la dificultad presunta de distintos problemas matemáticos.

\begin{figure}[H]
    \centering
    \includegraphics[width=0.8\textwidth]{imagenes/Cifrado_asimetrico.png}
    \caption{Ejemplo de cifrado asimétrico}
    \label{fig:cifrado_asimetrico}
\end{figure}

\subsection{Intercambio de claves Diffie–Hellman}\label{sec:diffie_hellman}
Como hemos mencionado anteriormente, el protocolo de Diffie-Hellman, publicado en 1976 \cite{diffie-hellman}, resuelve el problema de dos entidades que desean compartir una clave secreta para usar en un cifrado simétrico, pero su único canal de comunicación es inseguro y todo lo que intercambian lo observa el intruso. ¿Cómo pueden establecer una clave que el intruso no pueda deducir? La brillante idea de Diffie y Hellman fue aprovechar la dificultad del problema del logaritmo discreto en \(\mathbb{F}_p^*\). A continuación, se muestra el procedimiento del mismo:

\begin{enumerate}
  \item Alice y Bob acuerdan públicamente un primo grande \(p\) y un generador \(g\) de \(\mathbb{F}_p^*\).  
  \item Alice elige en secreto un entero \(a\) y calcula
    \[
      A \;\equiv\; g^a \pmod p.
    \]
  \item Bob elige en secreto un entero \(b\) y calcula
    \[
      B \;\equiv\; g^b \pmod p.
    \]
  \item Intercambian \(A\) y \(B\) por el canal público. Eve conoce \(p,g,A,B\).  
  \item Alice calcula la clave compartida
    \[
      K = B^a \equiv g^{ba} \pmod p,
    \]
    y Bob calcula
    \[
      K = A^b \equiv g^{ab} \pmod p.
    \]
\end{enumerate}

\begin{table}[H]
  \centering
  \begin{tabular}{p{4cm}p{8cm}}
    \toprule
    \textbf{Fase} & \textbf{Operación} \\
    \midrule
    Intercambio: elegir $p$ y $g$ & Un tercero de confianza elige y publica un primo \(p\) grande y un generador \(g\) de orden primo en \(\mathbb{F}_p^*\). \\
    Generación de claves secretas  & Alice elige \(a\), calcula \(A\equiv g^a\pmod p\); Bob elige \(b\), calcula \(B\equiv g^b\pmod p\). \\
    Intercambio público & Alice envía \(A\) a Bob, Bob envía \(B\) a Alice. \\
    Cálculos finales    & Alice calcula \(K\equiv B^a\pmod p\); Bob calcula \(K\equiv A^b\pmod p\). \\
    \bottomrule
  \end{tabular}
  \caption{Resumen del intercambio de claves Diffie–Hellman}
  \label{tab:dh-algoritmo}
\end{table}

\begin{ejemplo}
Supongamos que Alice y Bob acuerdan \(p=7\) y \(g=5\).  
Alice elige \(a=3\) y calcula
\[
  A = 5^3 \equiv 125 \equiv 6 \pmod 7,
\]
mientras que Bob elige \(b=4\) y obtiene
\[
  B = 5^4 \equiv 625 \equiv 2 \pmod 7.
\]
Tras intercambiar \(A=6\) y \(B=2\), ambos calculan
\[
  K = 2^3 \equiv 8 \equiv 1 \pmod 7,
  \quad
  K = 6^4 \equiv 1296 \equiv 1 \pmod 7,
\]
de modo que la clave compartida es \(K=1\).  
El intruso ve \(p,g,A,B\) pero, para calcular \(K=g^{ab}\), tendría que resolver el problema del logaritmo discreto en \(\mathbb{F}_7^*\), lo cual es prácticamente imposible cuando \(p\) tiene cientos de dígitos.
\end{ejemplo}

\subsection{Firma digital}
TODO: preguntar si es necesario esta seccion

\section{Resumen comparativo}
Los sistemas criptográficos actuales suelen combinar ambas: por ejemplo, se utiliza un esquema de clave pública para establecer una clave de sesión y luego un cifrado simétrico para procesar los datos.   

Hasta la fecha, la encriptación de clave pública es sustancialmente más lenta que la simétrica; sin embargo, no existe prueba de que deba serlo necesariamente. En la práctica, los puntos clave son:  
\begin{enumerate}
  \item La criptografía de clave pública facilita la firma eficiente, el no repudio) y la gestión de claves.  
  \item La criptografía de clave simétrica es muy eficiente para cifrar datos y para aplicaciones de integridad.  
\end{enumerate}

\paragraph{Observación (tamaños de clave)}  
Para lograr un nivel de seguridad equivalente, las claves privadas en sistemas de clave pública (por ejemplo, 2048 o 3072 bits en RSA) deben ser mucho más largas que las secretas en sistemas simétricos (por ejemplo, 128 bits en AES). Esto se debe a que, mientras que el ataque más eficiente contra la simétrica es la búsqueda exhaustiva, los sistemas de clave pública conocidos admiten algoritmos \emph{atajo} (por ejemplo, factorización) más eficientes que el barrido completo del espacio de claves. Por tanto, para igualar los 128 bits simétricos, se requieren claves RSA de unos 3072 bits, es decir, más de 24 veces más largas, aunque valores del orden de 10 veces mayores se citan frecuentemente en la literatura~\citep{SecurityStrengthRSA}.

\section{Principios de diseño criptográfico}

En 1883, Auguste Kerckhoffs~\cite{Petitcolas2011Kerckhoffs} formuló seis principios esenciales para el diseño de sistemas criptográficos:

\begin{enumerate}
  \item \textbf{Indescifrabilidad práctica:} El sistema debe ser, cuanto menos, imposible de romper en la práctica.
  \item \textbf{Seguridad sólo de la clave:} La seguridad no debe depender del secreto del algoritmo, sino únicamente de la clave.
  \item \textbf{Clave manejable:} La clave debe poder ser almacenada, pudiendo ser reemplazable.
  \item \textbf{Transmisión telegráfica:} El texto cifrado debe poder transmitirse sin errores por medios estándar.
  \item \textbf{Portabilidad:} Los documentos que implementan el sistema deben ser portátiles y operables por una sola persona.
  \item \textbf{Simplicidad de uso:} El sistema debe ser fácil de usar, sin requerir un esfuerzo mental excesivo.
\end{enumerate}

\section{Seguridad y modelos de ataque}
La seguridad de un sistema criptográfico se evalúa frente a distintos modelos de ataque, que describen el acceso y las capacidades del adversario. A grandes rasgos, estos modelos se agrupan en:

\subsection{Ataques pasivos y activos}

\begin{description}
  \item[Ataques pasivos:] El atacante únicamente observa el tráfico cifrado (o el sistema) sin modificarlo. Incluyen la intercepción de mensajes y el análisis de tráfico, en los que se extrae información de patrones de comunicación \cite{different_types_cryptography_attacks}\cite{active_and_passive_attacks_in_information_security}.
  \item[Ataques activos:] El atacante altera, inyecta o interfiere el flujo de datos. Ejemplos típicos son el man-in-the-middle (interposición entre emisor y receptor) y la modificación de mensajes en tránsito \cite{what_are_cryptographic_attacks}.
\end{description}

\subsection{Modelos clásicos de criptoanálisis}

\begin{itemize}
  \item \emph{Ataque por texto cifrado únicamente (COA):} El atacante dispone solo del texto cifrado. Es el caso más habitual porque en la práctica los atacantes a menudo solo logran capturar mensajes cifrados sin más información.
  \item \emph{Fuerza bruta (exhaustiva):} Se calculan y prueban todas las claves posibles hasta dar con la correcta. Su dificultad depende únicamente del tamaño de la clave \cite{cryptography_attacks_6_types_and_prevention}\cite{types_of_cryptographic_attacks_to_know}.
  \item \emph{Ataque con texto plano conocido (KPA):} El atacante dispone de algún texto original junto con su versión cifrada. Por ejemplo, sabe que un correo empieza con "Estimado cliente:" y ve esa misma parte cifrada. Con esas parejas (texto claro, texto cifrado) puede acortar la búsqueda de la clave.
  \item \emph{Ataque por texto plano elegido (CPA):} El atacante elige ciertos textos para cifrar y obtiene sus correspondientes cifrados. Es como pedirle al dueño de la caja fuerte que te abra cosas que tú le entregas. Con los pares que él devuelve, puedes analizar el comportamiento del sistema y encontrar debilidades.
  \item \emph{Ataque adaptativo por texto plano elegido (CPA2):} Variante de CPA donde el atacante en tiempo real elige un mensaje, ve su cifrado, y luego, basándose en lo que aprendió, decide el siguiente mensaje a cifrar. Y así sucesivamente, mejorando su estrategia con cada consulta.
  \item \emph{Ataque por texto cifrado elegido (CCA):} El atacante no solo puede cifrar, sino que también puede solicitar descifrar ciertos textos cifrados de su elección (salvo el que realmente le interesa). Con eso obtiene planos válidos que le ayudan a deducir información y, finalmente, descifrar el mensaje objetivo.
  \item \emph{Ataque adaptativo por texto cifrado elegido (CCA2):} Versión más poderosa de CCA, donde tras cada petición de descifrado, el atacante ajusta sus futuras preguntas para exprimir al máximo el sistema y extraer todos los datos posibles.
\end{itemize}

\subsection{Ataques avanzados}

\begin{itemize}
  \item \emph{Ataques de canal lateral (side-channel):} No atacan el cifrado matemático, sino que, se explota información derivada de la implementación (tiempos, consumo eléctrico, emisiones electromagnéticas) para reconstruir las claves sin romper directamente el algoritmo \cite{cryptanalysis_and_types_of_attacks}.
  \item \emph{Ataque de clave relacionada (related-key):} El atacante conoce cifrados de textos planos bajo claves que guardan alguna relación matemática con la clave objetivo (por ejemplo, variantes que difieren solo en unos bits). Si conoce cómo varía el cifrado al pasar de una clave a otra, puede deducir la clave original más rápido que con fuerza bruta.
  \item \emph{Ataque por sesgo estadístico (bias attacks):} Se buscan patrones no aleatorios en la generación de claves o estados intermedios para acotar posibles valores \cite{modern_cryptographic_attacks_guide_for_perplexed}. Si el generador de claves o el proceso intermedio tiende a preferir ciertos valores, se reduce drásticamente el número de opciones a probar.
  \item Ataque \emph{Evil Maid}: Consiste en acceso físico al dispositivo para instalar malware o keyloggers y robar claves cuando el usuario no está presente.
\end{itemize}