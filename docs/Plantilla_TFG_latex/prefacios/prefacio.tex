\chapter*{}
%\thispagestyle{empty}
%\cleardoublepage

%\thispagestyle{empty}

\input{portada/portada_2}



\cleardoublepage
\thispagestyle{empty}

\begin{center}
{\large\bfseries \myTitle}\\
\end{center}
\begin{center}
Leon Elliott Fuller\\
\end{center}

%\vspace{0.7cm}
\noindent{\textbf{Palabras clave}: curvas elípticas, criptografía, computación cuántica, ECDH, ECDSA, ECC, RSA}\\

\vspace{0.7cm}
\noindent{\textbf{Resumen}}\\
En 1985 Neal Koblitz y Victor Miller propusieron de forma independiente el uso de curvas elípticas en criptografía. Hoy en día, los esquemas basados en curvas elípticas (ECC) se han consolidado como una alternativa eficiente a RSA, pues ofrecen niveles de seguridad comparables con tamaños de clave sensiblemente menores. Esto los hace ideales para dispositivos con recursos limitados y para aplicaciones que requieren alto rendimiento.

En este trabajo profundizaremos en los fundamentos de ECC y describiremos los principales protocolos Elliptic Curve Diffie–Hellman (ECDH) y Elliptic Curve Digital Signature Algorithm (ECDSA). A continuación analizaremos tanto los ataques clásicos al problema del logaritmo discreto en curvas elípticas como las amenazas cuánticas, en particular el algoritmo de Shor, que podría comprometer ECC en la era de los ordenadores cuánticos.

Por último, se realizarán comparaciones de tiempo con respecto a la generación de claves entre RSA y ECC.

\cleardoublepage


\thispagestyle{empty}


\begin{center}
{\large\bfseries Theoretical foundations of elliptic curves and their application in cryptography}\\
\end{center}
\begin{center}
\myName\\
\end{center}

%\vspace{0.7cm}
\noindent{\textbf{Keywords}: elliptic curves, cryptography, quantum computing, ECDH, ECDSA, ECC, RSA}\\

\vspace{0.7cm}
\noindent{\textbf{Abstract}}\\
In 1985 Neal Koblitz and Victor Miller independently proposed the use of elliptic curves in cryptography. Today, elliptic curve schemes (ECC) have become a highly efficient alternative to RSA, offering comparable security with significantly smaller key sizes. This makes them ideal for resource-constrained devices and high-performance applications.

In this work we dive into the fundamentals of ECC and describe the main protocols used, such as Elliptic Curve Diffie–Hellman (ECDH) and Elliptic Curve Digital Signature Algorithm (ECDSA). We then analyze both classical attacks on the elliptic curve discrete logarithm problem and quantum threats, in particular Shor’s algorithm, which could compromise ECC in the era of quantum computers.

Finally, we present timing comparisons for key generation between RSA and ECC.

\chapter*{}
\thispagestyle{empty}

\noindent\rule[-1ex]{\textwidth}{2pt}\\[4.5ex]

Yo, \textbf{Leon Elliott Fuller}, alumno de la titulación Ingeniería Informática de la \textbf{Escuela Técnica Superior
de Ingenierías Informática y de Telecomunicación de la Universidad de Granada}, con NIE \myNIE, autorizo la ubicación de la siguiente copia de mi Trabajo Fin de Grado en la biblioteca del centro para que pueda ser consultada por las personas que lo deseen. Todo el código fuente así como este documento en formato LaTeX se puede encontrar en el siguiente repositorios de GitHub: https://github.com/leonfullxr/ECC-Thesis

\vspace{6cm}

\noindent Fdo: Leon Elliott Fuller

\vspace{2cm}

\begin{flushright}
Granada a X de mayo de 2025.
\end{flushright}


\chapter*{}
\thispagestyle{empty}

\noindent\rule[-1ex]{\textwidth}{2pt}\\[4.5ex]

D. \textbf{Jesús García Miranda}, Profesor del Área del Grado en Ingeniería Informática  del Departamento de Álgebra de la Universidad de Granada.

\vspace{0.5cm}

\vspace{0.5cm}

\textbf{Informa:}

\vspace{0.5cm}

Que el presente trabajo, titulado \textit{\textbf{\myTitle}},
ha sido realizado bajo su supervisión por \textbf{Leon Elliott Fuller}, y autorizamos la defensa de dicho trabajo ante el tribunal
que corresponda.

\vspace{0.5cm}

Y para que conste, expiden y firman el presente informe en Granada a X de mayo de 2025.

\vspace{1cm}

\textbf{El tutor:}

\vspace{5cm}

\noindent \textbf{Jesús García Miranda}

\chapter*{Agradecimientos}
\thispagestyle{empty}

       \vspace{1cm}


Antes que nada, quiero expresar mi más profundo agradecimiento a mis queridos padres, Allan y Milena, quienes han sido los pilares de apoyo en todo este recorrido. Sin su amor incondicional, su confianza y su esfuerzo diario, jamás habría llegado hasta aquí. Les estaré eternamente agradecido por regalarme la oportunidad de soñar en grande y de perseguir mis metas.

A continuación, quiero agradecer a mis otros dos pilares, Pablo Balbuena Fernández e Iñigo Zapirain Laboa, dos personas que me habéis apoyado y empujado siempre a por más, a por lo imposible. Me habéis enseñado e inculcado valores y disciplinas que nunca hubiera descubierto por mi cuenta. Gracias por haberme llevado al límite y por mantenerme con los pies en la tierra cuando el camino se ponía cuesta arriba (y menudas cuestas hemos tenido que subir).

Quiero dar también, un fiel agradecimiento a mis compañeros de clase, concretamente a Toni de Potocolom, quien ha sido el fruto de mi camino espiritual e intelectual. Gracias a ti he podido abarcar mucho conocimiento y me has enseñado a mirar el mundo con nuevos ojos. Gracias también a Karim por haberme aguantado todos estos años de carrera, la primera persona con la que tuve contacto nada más pisar la ETSIIT.

Por último, me gustaría agradecer a mi tutor Jesús García Miranda, una persona más que un simple tutor o profesor. Eres una persona con un gran corazón y todos los alumnos que han podido tener el placer de haberte tenido de profesor o haber coincidido contigo, estoy seguro de que le has dejado un recuerdo bonito. Gracias por haberme apoyado y por haberme ayudado en la elaboración de este gran proyecto final que depara mi etapa universitaria.

A todos vosotros: gracias de verdad, de corazón, por formar parte de esta historia.