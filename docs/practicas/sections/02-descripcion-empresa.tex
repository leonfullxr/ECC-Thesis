\chapter{Descripción de la empresa}

La empresa donde se realizaron las prácticas se llama \textbf{Wazuh, Inc.}, constituida como Delaware corporation con EIN 47-2523953 y su sede central situada en 1999 S. Bascom Ave, Suite 700 PMB\#727, Campbell, CA 95008, Estados Unidos \cite{wazuh_support_agreement}. 

Wazuh fue fundada en 2015 por Santiago Basset como una bifurcación del proyecto OSSEC, consolidándose rápidamente como una de las plataformas de seguridad de código abierto más relevantes del mercado \cite{wazuh_wikipedia_es}. 

Su equipo está formado por más de 200 profesionales distribuidos por todo el mundo, lo que refleja su expansión y alcance global \cite{wazuh_linkedin}. 

Según la propia compañía, Wazuh protege más de 15 millones de endpoints y da servicio a más de 100 000 usuarios empresariales, ofreciendo soluciones SIEM y XDR de código abierto \cite{wazuh_homepage}. 

La actividad principal de Wazuh consiste en el desarrollo de una plataforma de detección de amenazas y respuesta a incidentes, que integra gestión de logs, monitorización de integridad de ficheros y respuestas activas \cite{wazuh_homepage}. Su agente multiplataforma soporta sistemas operativos como Linux, Windows y macOS \cite{wazuh_agent_installation}, e incorpora integraciones para contenedores Docker, Kubernetes, Ubuntu, CentOS y entornos en la nube, garantizando el cumplimiento de normativas como PCI DSS o HIPAA \cite{wazuh_regulatory_compliance}. 

Wazuh cuenta con representación en España a través de partners oficiales y equipos remotos, con actividad en ciudades como Madrid y Granada \cite{wazuh_partners}. 

Según la plataforma Tracxn, Wazuh es considerado un startup de ciberseguridad especializado en HIDS, y provee servicios profesionales y soporte a organizaciones de muy diverso tamaño \cite{wazuh_tracxn}. 

Gartner reconoce a Wazuh como una plataforma de seguridad open source sólida y escalable en entornos on-premise y cloud \cite{wazuh_gartner}. 

El personal cualificado de Wazuh incluye ingenieros de seguridad, desarrolladores y analistas, especializados en ciberseguridad, cumplimiento normativo y tecnologías cloud \cite{wazuh_linkedin}. 

En cuanto a dotación tecnológica, la infraestructura de Wazuh está compuesta por componentes como el Wazuh Manager, el Wazuh Indexer y el Wazuh Dashboard, que se integran con Elastic Stack para el análisis y la visualización de grandes volúmenes de datos en tiempo real \cite{wazuh_documentation}. Esta arquitectura centralizada y escalable permite monitorizar y administrar miles de agentes desde un único panel de control.

\chapter{Descripción de la empresa}

La empresa Wazuh, Inc., con sede central en Campbell, California (1999 S Bascom Ave, Campbell, CA 95008, Estados Unidos) \cite{linkedin_wazuh}\cite{slideshare_wazuh_bassett}, cuenta además con una oficina principal en Granada, España \cite{glassdoor_wazuh_granada}. Fundada en 2015, Wazuh es una compañía privada dedicada al desarrollo de soluciones de ciberseguridad de código abierto \cite{wazuh_about_us}.

\section{Actividad laboral}
Wazuh ofrece una plataforma unificada de XDR y SIEM para la protección de endpoints y cargas de trabajo en la nube \cite{wazuh_homepage}. Sus funciones principales incluyen la gestión de logs, la monitorización de integridad de ficheros y la detección de vulnerabilidades \cite{wazuh_homepage}\cite{wazuh_about_us}. Adicionalmente, Wazuh proporciona módulos de cumplimiento normativo para estándares como PCI DSS e HIPAA \cite{wazuh_regulatory_compliance}. La plataforma soporta integraciones nativas con entornos Docker y Amazon Web Services (AWS) para garantizar una monitorización continua y centralizada de infraestructuras heterogéneas \cite{wazuh_agent_installation}.

\section{Personal cualificado}
La plantilla de Wazuh está entre los 200 y 500 empleados a nivel mundial, entre ingenieros de software, ingenieros de seguridad (soporte técnico), ingenieros en Cloud, ingenieros de QA, desarrolladores de C/C++, diseñadores UX/UI, especialistas en DevOps y expertos en ciberseguridad \cite{linkedin_wazuh}\cite{wazuh_wikipedia_es}.

\section{Dotación tecnológica}
La infraestructura de Wazuh se despliega completamente en Amazon Web Services (AWS), haciendo uso de instancias EC2, almacenamiento en S3 y servicios de monitorización como CloudWatch para procesar y analizar datos de seguridad en tiempo real \cite{wazuh_agent_installation}. En cuanto a cualquier otra dotación tecnológica como servidores, almacenamiento, o clústeres de computo todo está en la nube contratado con Amazon Web Services (AWS). AWS es una plataforma de computación en la nube que ofrece más de 200 servicios como infraestructuras como servicio (IaaS), plataforma como servicio
(PaaS) y software como servicio (SaaS). Estos servicios incluyen potencia de cómputo, opciones de almacenamiento, capacidad de red, etc.
Para garantizar el cumplimiento de normativas como PCI DSS e HIPAA, Wazuh integra módulos específicos de compliance que se ejecutan dentro de esa misma infraestructura en la nube \cite{wazuh_regulatory_compliance}. No se asignó equipamiento de oficina presencial en la sede de Granada debido al empleo de modelos de trabajo remoto.
Por último, para la asignación de tarea se trabaja con Jira. Es una herramienta de gestión de proyectos y seguimiento de tareas que ayuda a los equipos a planificar, rastrear y gestionar proyectos de desarrollo de software.
