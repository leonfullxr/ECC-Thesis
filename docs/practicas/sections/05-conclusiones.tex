\chapter{Conclusiones}
Durante estos tres meses de prácticas en Wazuh he tenido la oportunidad de colaborar con equipos multidisciplinares y entender en profundidad la organización y el funcionamiento de una empresa tecnológica de gran envergadura. Este periodo me ha permitido adquirir una visión amplia tanto del ámbito corporativo como del técnico, y experimentar de primera mano cómo se gestionan proyectos complejos en un entorno real. He aprendido a trabajar con herramientas y tecnologías punteras en ciberseguridad, tales como Kubernetes (EKS y entornos autogestionados), Terraform y Ansible para la provisión y configuración de infraestructura en AWS y OpenSearch/Elasticsearch para la indexación y análisis de logs. Asimismo, reforcé mis conocimientos sobre el despliegue de Wazuh Manager y Wazuh Agent en arquitecturas de alta disponibilidad, manejo de volúmenes persistentes y certificados TLS en entornos distribuidos. Gracias al soporte continuo de la comunidad open source de Wazuh (Reddit, Discord, Slack, GitHub), pude consultar y resolver rápidamente dudas, lo que facilitó enormemente la integración de buenas prácticas y soluciones ya contrastadas.

Por otro lado, he mejorado significativamente mi comprensión de los conceptos de redes y protocolos: aprendí a diagnosticar problemas con \texttt{tcpdump}, interpretar el intercambio de paquetes TCP (SYN, SYN–ACK, RST), configurar reglas de \texttt{iptables} para permitir tráfico SMTP y depurar conexiones SMTP/TLS con Postfix. Asimismo, profundicé en el uso de APIs REST de Wazuh, autenticación con tokens JWT, scripting avanzado en Bash (manipulación de JSON con \texttt{jq}, construcción dinámica de HTML/CSS, manejo de fechas y registros), y configuración de \texttt{cron} para automatizar tareas diarias. Esta experiencia global me ha dotado de un conjunto de competencias técnicas sólidas en administración de sistemas Linux, orquestación de contenedores, herramientas de automatización, gestión de seguridad en la nube y metodologías de diagnóstico y monitorización, que sin duda fortalecerán mi perfil profesional y me preparan para afrontar nuevos retos en el ámbito de la ciberseguridad y la ingeniería de infraestructura.