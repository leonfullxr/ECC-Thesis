\chapter{Introducción}

La realización de prácticas profesionales en una empresa representa una experiencia esencial para la formación académica del alumno, ya que facilita la puesta en práctica de los saberes adquiridos durante la carrera en un contexto real y contribuye al desarrollo de destrezas y competencias valiosas.

El propósito de esta memoria es mostrar con detalle las tareas ejecutadas, los retos afrontados y las soluciones puestas en marcha a lo largo del periodo de prácticas en Wazuh, compañía referente en soluciones de seguridad de código abierto fundada en 2015 con sede en Campbell, California \cite{wazuh_about_us}. Las prácticas se desarrollaron entre marzo y mayo de 2025, trabajando principalmente desde la oficina de Granada y en remoto.

Durante este periodo, se expondrán de manera pormenorizada las tecnologías utilizadas (Kubernetes, AWS, Docker, scripting en Bash), los obstáculos encontrados (problemas de despliegue, configuraciones TLS, firewall interno, etc.) y las estrategias adoptadas, así como su vinculación con las asignaturas relevantes del plan de estudios.

A lo largo del documento, la memoria se estructura de la siguiente manera:
\begin{itemize}
  \item \textbf{Descripción de la empresa}: se contextualiza el entorno de Wazuh, su actividad principal, el equipo humano y la dotación tecnológica.  
  \item \textbf{Trabajo realizado}: se detallan las tareas llevadas a cabo, los problemas técnicos planteados, las herramientas y soluciones implementadas, indicando qué asignaturas sustentan cada técnica.  
  \item \textbf{Valoración personal}: se reflexiona sobre la experiencia adquirida, las competencias desarrolladas y el impacto profesional de este periodo.
\end{itemize}