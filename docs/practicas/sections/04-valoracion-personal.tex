\chapter{Valoración Personal}

\section{Asignaturas Relacionadas}
Para la ejecución de las tareas anteriores, resultaron fundamentales los conocimientos de las siguientes asignaturas del plan de estudios:
\begin{itemize}
  \item \textbf{Sistemas Operativos}: Configuración avanzada de Linux y Windows, administración de servicios, scripting en Bash y PowerShell.  
  \item \textbf{Ingeniería de Servidores}: Fundamentos de criptografía, detección de intrusiones, gestión de incidentes y aplicación de estándares de seguridad (PCI DSS, HIPAA).  
  \item \textbf{Fundamentos de Redes}: Protocolos TCP/IP, configuración de firewalls, conceptos de IDS/IPS y análisis de tráfico con Wireshark/Suricata.  
  \item \textbf{Fundamentos de Bases de Datos}: Ingesta y consulta de grandes volúmenes de datos (alertas) en Elasticsearch/OpenSearch, optimización de índices y consultas.  
  \item \textbf{Ingeniería de Software}: Diseño de arquitecturas distribuidas, uso de herramientas de control de versiones (Git) y metodologías ágiles en proyectos de DevOps.  
\end{itemize}

\section{Valoración de la Experiencia}
Durante el periodo de prácticas en Wazuh, pude adquirir competencias prácticas en despliegue y administración de soluciones SIEM/XDR, así como en el desarrollo e implementación de reglas de detección basadas en estándares de ciberseguridad. La exposición continua a entornos empresariales reales y la colaboración con profesionales de distintas áreas fortalecieron mi capacidad para trabajar bajo presión y resolver problemas complejos de manera eficiente.

He de destacar que entré en un equipo donde no había un jefe, es decir, en cada equipo de zona horaria hay un jefe del mismo, que se encarga de organizar, vigilar y ayudar a sus subordinados. Yo tuve que reportar directamente al jefe de todo el equipo de Operaciones. Esto hizo que mi recorrido sea más difícil, no obstante, tuve la ayuda necesaria y pude completar todos los obstáculos y retos que me encontré por delante.